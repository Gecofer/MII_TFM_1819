% Introducción general del área del conocimiento a la que el tema elegido está vinculado.

% Esta parte es importante para contextualizar el contenido de la obra. Este punto debe de dar evidencias o justificar la esencia o el espíritu del trabajo realizado. Este apartado no es un resumen del trabajo. Es más filosófico y personal que técnico. De alguna manera habla del autor y de lo que pretendía cuando comenzó este trabajo. Esta parte no suele ser muy extensa. Hay que exponer el problema global de manera tan simple como se pueda. Presentar una visión más amplia, más holística del problema. No sobrestimar la familiaridad del lector con el tema del trabajo. No todos los lectores son especialistas en la materia ni la memoria debe de estar redactada para estos. Ayudaría imaginar a una persona, profesor, profesional... que se desenvuelve en un área diferente aunque tenga conocimientos generales técnicos propios de la titulación que se ha cursado. Esta persona es inteligente, tiene su mismo nivel de conocimiento general, pero sabe poco de la literatura, jerga o los trucos que se refieren a su tema particular.

% Escribir de manera que interese vivamente al lector a continuar leyendo. Para los primeros párrafos, la tradición permite la prosa, que es menos dura que el rigor exigido por la escritura científica. Es una buena idea preguntar a alguien que no sea un especialista sobre lo que opina tras leer la memoria. ¿Es una introducción adecuada? ¿Es fácil de seguir? ¿Es interesante?

% Indicar los motivos que han llevado a realizar el TFM y en concreto uno de esta temática. Si había alternativas, hay que valorarlas y justificar las razones de esta decisión. Se debe dejar claro cuál es el tema y porqué es importante.


\chapter{Introducción}
\label{ch:introduccion}

\begin{quote}
  {\bf\textsc{Resumen:}} Este capítulo define los límites del proyecto: problema a resolver, motivos de su elección, objetivos previstos y estructura. Teniendo como propósito convencer al lector para que no aparte la vista del trabajo.
\end{quote}


\section{Introducción}

El trabajo que se presenta a continuación es un estudio ...

% Hablar sobre que vamos a estudiar

% Que relevancia tienen los SIG aquí



\section{Definición del problema}

% Dedique este tema a aclarar lo que pretende de hecho con su esfuerzo de investigación. El problema es la cuestión a ser respondida por el trabajo, que motivó su realización. Es una cuestión que ya se formó en su mente, derivada de teorías del área investigada y de su observación sobre un fenómeno. Normalmente se utilizan los siguientes subíndenes como medios para determinar claramente los objetivos, lo que también colabora para la delimitación del alcance del trabajo. Está estrechamente vinculado al objetivo general, que normalmente consiste en encontrar la respuesta al problema de investigación.

% ¿Qué has visto que es un problema que necesita solución? ¿Es viable? ¿Puedes hacer? El problema es siempre una dificultad, una laguna.

% Breve descripción antes de poner el objettivo de manera clara y concisa

El trabajo tiene como objetivo la integración de Información Geográfica en la Web Semántica ...



\section{Motivación}

% Uno de los debates más habituales en la utilización de los Sistemas de Información Geográfica en entornos académicos es hasta que punto implican realmente un avance científico, con lo que su presencia como asignatura sería de pleno derecho, o si son sólo una herramienta, poco más que un procesador de textos, que los alumnos deberían aprender por su cuenta y que, como mucho, podría servir para explicar visualmente otros contenidos.

\textit{?`Qué motivación y razones me llevan a comenzar un proyecto como este?} Cómo Ingeniera Técnica Informática y estudiante del Máster Universitario en Ingeniería Informática, he adquirido a lo largo del grado y del máster conocimientos y usos muy diversos que se le puede dar a la tecnología. Después de haber realizado el Trabajo de Fin de Grado en \textit{Análisis de Información Geográfica mediante QGIS y R centrado en el Modelo Digital del Terreno}, he querido seguir cumpliendo el objetivo que me propuse: conocer más el mundo en el que se mueven los SIG, puesto que en estos dos años me es imposible abarcar todas las áreas relacionadas con el mismo. Además, destacar que durante la realización del Máster Universitario en Ingeniería Informática, he adquirido algunos conocimientos en la temática de Web Semántica a partir de la asignatura de \textit{Desarrollo de Software basado en Componentes y Servicios} impartida por Manuel Ignacio Capel Tuñón. Es más, las ganas de adentrarme en este mundo, ha hecho que durante este año haya asistido al curso de Web Semántica del Máster de Desarollo de Software, impartido por el tutor junto con otros profesores. Por todo esto y más, y con la idea de pasar tanto en lo personal como en lo profesional a un siguiente nivel mi Trabajo de Fin de Grado, surgió este Trabajo de Fin de Máster en el que intento dar cabida a la Web Semántica Geoespacial, con el fin de enriquecerme tanto personal como profesionalmente y de adentrar al lector en este tema tan interesante.\\ 

% el profesor (actual tutor) me introdujo en un ámbito que desconocía y del que apenas había oído hablar. 

% hablar sobre los cursos asistidos este año
% hablar sobre la conferencia que fui en Madrid
% hablar sobre la asignatura de SIG en el grado
% hablar sobre la asignatura de Capel

La idea de este proyecto comenzó a tomar forma a principios de curso, tras mantener varias reuniones con el tutor del TFG y ver que posibilidades había de seguir estudiando y ampliando los Sistemas de Información Geográficos de manera más práctica y profesional. Entonces, partiendo de todos los conocimientos que tengo a priori de la realización de este trabajo, me pregunto lo siguiente: \textit{?`existen interoperabilidad en los Sistemas de Información Geográfico?}. Este tipo de preguntas ha llevado a cabo el desarrollo del análisis en donde, gracias a ... En los sucesivos capítulos, iremos detallando cada uno de los conceptos que aquí mencionamos.


% coursera

En este curso verá en que consisten las tecnologías de la Web Semántica y como se utilizan en la web actual. También tendrá la oportunidad de realizar varios proyectos aplicando todas estas tecnologías para resolver los problemas que Marcelo ha comentado anteriormente. Por ejemplo, ¿le gustaría que Google entendiera cada uno de los componentes de su página web? O si usted tienen una tienda virtual, ¿le gustaría que Google fuera capaz de identificar los distintos productos que forman parte de su tienda virtual y los desplegara al momento de hacer una búsqueda? Esto se consigue usando [DESCONOCIDO] y les mostraremos como se relaciona en este curso con las tecnologías de la Web Semántica.

Además, ¿le gustaría poder acceder a la Wikipedia como si fuera una tabla Excel? ¿O le gustaría poder conocer como han hecho los gobiernos para hacer públicos sus datos y de esta manera permitir que los ciudadanos puedan acceder a la información de como se gastan sus impuestos, o puedan entender de una manera más sencilla como le afecta una ley? ¿O le gustaría saber como hacen los biólogos para compartir sus datos en la web?


\section{Premisas e Hipótesis}

% La mejor forma de determinar el tema abordado es a través de premisas e hipótesis. La hipótesis consiste en una afirmación que usted considera verdadera y que va a probar o buscar probar a lo largo de su trabajo. Otra forma es delimitar el problema en forma de una pregunta de partida. Las hipótesis presentadas aquí son probadas en su trabajo es lo que llamas de tesis.

La realización de este trabajo ha partido de varios supuestos básicos: ... Por último, considero que este trabajo ofrece posibilidades que merecen la máxima difusión, que son desconocidas para gran parte del público en general y de los Ingenieros en Informática en particular.


\section{Objetivos}

% Es la respuesta al problema especificado anteriormente, es decir, lo que se pretende hacer y que, después de alcanzado, habrá terminado el trabajo. Algunos verbos utilizados para determinar el objetivo general: contribuir / facilitar / subsidiar / proponer / aclarar / permitir / agregar / comprender.

Este apartado recoge los objetivos iniciales marcados para la realización del trabajo, especificando los propósitos que se esperan conseguir del mismo. A continuación, se detallan tanto los objetivos generales como específicos para el TFM.

\subsection{Objetivo General}

El objetivo de este proyecto es profundizar en ...

\subsection{Objetivos Específicos}

% Los objetivos específicos detallan los objetivos generales a través de etapas o fases de investigación. Se deben utilizar verbos en el infinitivo, señalando las acciones propuestas para alcanzar el objetivo general. Los verbos utilizados aquí son los de acción, que serán utilizados en la metodología.

% Comprobar si los objetivos puestos del principio corresponden con los objetivos que he cumplido al terminar el trabajo

En el subapartado anterior, se mencionan a grandes rasgos los objetivos que vamos a lograr con dicho análisis y aplicación. Por tanto, detallaremos los objetivos más específicos del proyecto, en donde al final del documento comprobaremos que dichos objetivos se han cumplido:

\begin{enumerate}
	
	\item 
	
	%\item \textbf{Análisis exhaustivo de los MDT}, contemplando las posibilidades que presentan para su futura aplicación en el ámbito de los SIG.
	
	%\item Aprender a \textbf{manipular información del terreno} mediante el empleo de herramientas como QGIS y R.
	
	%\item Apreciar las carencias o mejoras que supone el \textbf{uso de R frente a QGIS}.
	
	%\item Analizar un MDT con diversos paquetes de R y en consecuencia, \textbf{crear un paquete en R} que encapsule el análisis de sombras.
	
	%\item \textbf{Conocer diversas fuentes de información}, con las que obtener datos precisos del terreno y de buena calidad.
	
	%\item Mostrar algunos \textbf{nuevos métodos de trabajo}, estrechamente adaptados al tratamiento digital de la información y actualmente en rápido desarrollo. % Para ello, se expondrán las bases conceptuales, así como métodos de construcción y tratamiento de los modelos digitales del terreno (MDT), un caso de enorme interés dentro del conjunto de la cartografía digital.
	
	%\item \textbf{Aportaciones a la comunidad o al lector}, para que el proyecto sirva como puerta de acceso al mundo de los SIG y facilite el acceso a parte de los conocimientos actuales disponibles.
	
	%\item \textbf{Aportaciones hacia mi persona} en la adquisición de conocimientos SIG, manejo de herramientas como QGIS y R, análisis de MDT para mapas de sombras y aprendizaje para la creación de una librería en R.

\end{enumerate}


\section{Estructura de la monografía}

% En este ítem usted describirá cómo está constituida la monografía, indicando lo que será encontrado en cada una de las sesiones siguientes.

En este primer capítulo, se ha desarrollado una pequeña introducción al contexto que desarrolla este trabajo, puntualizando en los motivos de la elección del mismo. A continuación, se detallan de forma resumida los contenidos del resto de capítulos de este documento:

\begin{itemize}
	\item En el \textit{capítulo 2} \textbf{(Sistemas de Información Geográfica)}, se contextualiza el trabajo, presentando para ello los SIG. Además, veremos brevemente el estado del arte sobre ciertos aspectos relacionados con el proyecto.
	
	\item En el \textit{capítulo 3} \textbf{(Web Semántica)}, se expone el concepto de Web Semántica, necesario para comprender su aplicación en el \texttt{capítulo 4}.
	
	\item En el \textit{capítulo 4} \textbf{(Aplicación)}, 
	
	\item En el \textit{capítulo 5} \textbf{(Resultados)}, 

	\item En el \textit{capítulo 6} \textbf{(Conclusiones y Trabajo Futuro)}, se evalúan todas las propuestas realizadas en los \texttt{capítulos \ref{ch:aplicacion}} y \texttt{\ref{ch:resultados}}, recopilando tanto lo que se ha hecho a lo largo del desarrollo de este trabajo como las conclusiones y resultados finales obtenidos de esta experiencia. Se comentan además una serie de mejoras para el futuro que se podrían aplicar, pero que por diversos motivos caen fuera del ámbito de este trabajo.
	
	%\item Por último se incluye un \textbf{(Apéndice)}, que incorpora material adicional que complementa la información de algunos de los capítulos de este documento.
	
\end{itemize}


%\begin{figure}[H]
%	\centering
%	\includegraphics[height=4.5cm]{imagenes/4_solsticio-equinoccio.jpg}
%	\caption{Solsticios y Equinoccios para las 12 a.m.}
%	\label{fig:solsticio-equinoccio}
%\end{figure}