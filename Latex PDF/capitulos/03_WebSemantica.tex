% Aquí contendrán los métodos y procedimientos adoptados en el desarrollo del trabajo. Esta es una de las sesiones más importantes pues demuestra el poder científico que se utilizó para la investigación. Sin una buena metodología la investigación puede perder la validez. El investigador debe utilizar métodos o técnicas aceptadas por la comunidad científica en la búsqueda de probar sus hipótesis.

% La metodología elegida debe ser aquella que más se adecue a su objeto de estudio y al enfoque aplicado. Hay dos métodos principales: 1) cuantitativo, que es el uso de instrumental estadístico, de datos numéricos; y 2) cualitativo, que se caracteriza por la calificación de los datos recogidos, durante el análisis del problema.


\chapter{Web Semántica}
\label{ch:web-semantinca}

\begin{quote}
  {\bf\textsc{Resumen:}} Este capítulo ...
\end{quote}

\section{Web Semántica}

\subsection{Web actual vs. Web Semántica}

% mencionar algo sobre lo de interoperabilidad

% hablar sobre OGC

\subsection{Lenguaje de marcado geográfico}

En literatura, se han desarrollado varios estándares útiles para facilitar el intercambio de datos espaciales. Por ejemplo, el archivo de datos geográficos (GDF) es un formato de archivo intercambiable para compartir datos geográficos. GDF está diseñado específicamente para el intercambio de datos espaciales para sistemas de transporte inteligente (ITS).

% aquí se debe de hablar de GML

% hablar sobre GDF y SDTS

\section{XML}

Geography Markup Language (GML) es “una gramática XML escrita en XML Schema para el modelado, transporte y almacenamiento de información geográfica, incluidas las propiedades espaciales y no espaciales de las características geográficas”

\section{SVG}

Como su nombre lo indica, SVG es un gráfico vectorial, que es diferente de los formatos de imagen rasterizada, como GIF, JPEG y PNG. Como gráfico vectorial, SVG utiliza declaraciones matemáticas para describir las formas y los trazados de una imagen.\\

% hablar sobre la escalibilidad de SVG

Los datos GML se pueden transformar en formato SVG utilizando un procesador XSLT a través de la combinación con una hoja de estilo. XSLT (Transformaciones de lenguaje extensible de hojas de estilo) es un lenguaje para transformar documentos XML en otros documentos XML u otros objetos, como HTML para páginas web.

% limitaciones de SVG

\section{RDF}

\section{Ontologías}

\subsection{OWL}


\section{Servicios web geoespaciales}

Con el desarrollo de estándares abiertos, surgieron servicios web para la interoperabilidad de datos a través de la web. Los servicios web son componentes de software autocontenidos y autodescritos que pueden ser descubiertos e invocados por otros componentes de software a través de la web.


\section{Resumen del capítulo}

- Interoperabilidad de datos geoespaciales: integración y estandarización

- Dan problemas GDF y SDTS

- Ventajas del XML

- ¿Por qué usar XML?

- ¿Por qué es escalable SVG?

- Limitaciones SVG

Este capítulo presenta la información de fondo sobre la interoperabilidad de datos geoespaciales y las tecnologías más avanzadas para lograr la interoperabilidad de datos geoespaciales, como GML, SVG y servicios web geoespaciales. Aunque se han desarrollado muchas bases de datos de SIG, la interoperabilidad de los datos geoespaciales sigue siendo un desafío para la comunidad geoespacial. GML como formato estándar de intercambio de datos tiene como objetivo lograr el objetivo de la interoperabilidad de los datos al proporcionar mecanismos para compartir y reutilizar datos a nivel de características en la Web. Sin embargo, GML se ha diseñado para mantener el principio de separar el contenido de la presentación. Por lo tanto, se puede usar SVG para diseñar los datos GML para la presentación. Como gráfico vectorial, SVG puede mostrar mapas de alta calidad. Mientras que GML proporciona un medio para codificar y transportar características geoespaciales a XML, SVG proporciona un medio para mostrar estas características geoespaciales codificadas en GML en mapas vectoriales en la Web. Un tema de preocupación es cómo realizar colas y extraer características de la base de datos para responder a las solicitudes de los usuarios. Las especificaciones de implementación de servicios web geoespaciales desarrolladas por OGC cumplen esta función. Específicamente, (1) Web Feature Service (WFS) permite a un cliente recuperar, consultar y manipular datos geoespaciales a nivel de característica codificados en GML (Geography Markup Language) desde múltiples fuentes; (2) Web Map Service es capaz de crear y mostrar mapas que provienen simultáneamente de múltiples fuentes heterogéneas en un formato de imagen estándar; (3) el Servicio de cobertura web proporciona acceso a conjuntos potencialmente detallados y ricos de información geoespacial en formas que son útiles para la representación del lado del cliente, cobertura multivalor y aportes a modelos científicos y otros clientes; (4) Web Processing Service define reglas para estandarizar entradas y salidas (solicitudes y respuestas) de servicios de procesamiento geoespacial; (5) El Servicio de catálogos proporciona catálogos para servicios web de OGC y admite la capacidad de publicar y buscar colecciones de información descriptiva (metadatos) para datos, servicios y objetos de información relacionados. La arquitectura que hace uso completo de los servicios de catálogo web y otros servicios web, como OGC WFS, WMS, WCS y WPS, se denomina Arquitectura Orientada a Servicios (SOA). La SOA se aleja de los sistemas monolíticos hacia sistemas distributivos con componentes interoperables, y las implementaciones de la SOA pueden disminuir los problemas en la duplicación y el mantenimiento de los datos y modelos.











