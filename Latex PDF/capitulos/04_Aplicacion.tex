\chapter{Aplicación}
\label{ch:aplicacion}

\begin{quote}
	{\bf\textsc{Resumen:}} Este capítulo ...
\end{quote}



Un SIG ha sido ampliamente utilizado por una variedad de aplicaciones, muchas bases de datos geográficas han sido desarrolladas por diferentes programas y software. Sin embargo, sigue siendo un gran problema compartir estos datos geoespaciales y usarlos para el desarrollo de aplicaciones SIG. No es que los datos no estén disponibles, hay una gran cantidad de datos geográficos almacenados en diferentes lugares y en diferentes formatos, pero la reutilización de datos para nuevas aplicaciones y el intercambio de datos son tareas abrumadoras debido a la heterogeneidad de los sistemas existentes en términos de conceptos de modelado de datos, técnicas de codificación de datos y estructuras de almacenamiento, etc.\\

Hay dos problemas que resultan directamente de la no interoperabilidad de las bases de datos. Uno es el cambio en la exactitud de los datos. Después de que los datos se conviertan de un formato a otro, pueden ocurrir problemas como la precisión de coordenadas, errores de omisión, nombres de atributos faltantes o incorrectos y una topología incorrecta. El segundo problema es la inversión de tiempo y dinero para la conversión de datos. Se ha desperdiciado mucho dinero y tiempo en la conversión de datos o en el desarrollo de herramientas de conversión de datos.\\

La interoperabilidad de datos significa la capacidad de utilizar una variedad de formatos de datos. Los datos geoespaciales interoperables pueden ser utilizados por diferentes tipos de programas y aplicaciones. Con datos geoespaciales interoperables, los usuarios deben poder solicitar, acceder e integrar datos fácilmente sin importar dónde se almacenen los datos (local o remotamente). La interoperabilidad de los datos geoespaciales es extremadamente importante para las aplicaciones geoespaciales, ya que existían grandes cantidades de datos espaciales de diferentes formatos geográficos y hay una mayor demanda de reutilización de estos datos espaciales existentes para la toma de decisiones. La interoperabilidad de los datos geoespaciales elimina las barreras para el intercambio de datos y permite a los usuarios acceder, mapear, visualizar y analizar directamente datos con diferentes formatos de datos espaciales. Los datos geoespaciales interoperables hacen posible la distribución rápida de información y el intercambio entre departamentos.\\




\section{Resumen del capítulo}





