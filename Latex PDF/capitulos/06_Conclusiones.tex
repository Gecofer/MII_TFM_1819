% La conclusión debe contener los principales aspectos y contribuciones para finalizar el trabajo presentado. Se debe presentar lo que se esperaba del trabajo a través de los objetivos introducidos inicialmente y mostrar lo que se logró. No se debe insertar un nuevo asunto en la conclusión. Aquí el autor presentará las propias impresiones sobre el trabajo efectuado. Es importante también que se identifiquen limitaciones y problemas que surgieron durante el desarrollo del trabajo y cuáles las consecuencias del mismo.

% Los trabajos futuros deben contener oportunidades de expansión del trabajo presentado, así como nuevos proyectos que pudieron ser vislumbrados a partir del desarrollo del trabajo.

\chapter{Conclusiones y Trabajo Futuro}
\label{ch:Conclusiones y Trabajo Futuro}

\begin{quote}
  {\bf\textsc{Resumen:}} Este capítulo recoge los principales aspectos y contribuciones obtenidos para la integración de Información Geográfica en la Web Semántica. De igual manera, comprobaremos el cumplimiento de los objetivos que introducimos inicialmente en el \texttt{Capítulo 1}. Asimismo, se comentarán posibles propuestas de mejora de cara a sucesivas iteraciones del proyecto, además de comentar otras posibles aplicaciones empleadas, que caen fuera del ámbito de este trabajo.
\end{quote}


\section{Conclusiones}

En este TFM se ha profundizado ... \\


Por tanto, mediante el desarrollo de este TFM, considero que se han conseguido los objetivos señalados en el capítulo 1, en concreto:

\begin{enumerate}
	
	\item 
	
	%\item Se ha profundizado en el uso de los MDT mostrando las posibilidades que ofrecen.
	%\item Se ha mostrado cómo manipular Información Geográfica mediante QGIS y R.
	%\item Se ha realizado una comparación entre QGIS y R para el análisis de sombras a partir de MDT.
	%\item Se han encapsulado las funciones utilizadas mediante el desarrollo de un paquete R.
	%\item Se han estudiado diversas fuentes de Información Geográfica explicando criterios para su utilización más adecuada.
	%\item Se ha mostrado, mediante ejemplos, métodos de trabajo sobre la Información Geográfica en formato digital.
	%\item Se ha prestado especial atención a desarrollar en un trabajo autocontenido, tratando de que sea accesible a cualquier persona interesada en el tema.
	%\item He adquirido experiencia en el trabajo sobre SIG, en concreto sobre QGIS y R. He profundizado en el análisis de MDT. He aprendido a desarrollar librerías en R. He trabajado con numerosa información procedente de distintas fuentes para especializarme en los temas tratados, seleccionando la más adecuada en cada caso.
\end{enumerate}



La WS es aún una visión, un proyecto de futuro muy ambicioso, que permitirá, con ayuda de la Inteligencia Artificial, realizar un sinfín de operaciones en la Web, mucho más amplias que las ofertadas hoy en día. El tener toda la información etiquetada sintáctica y semánticamente facilitará la implementación eficaz de los llamados agentes inteligentes, capaces de ofrecer información Web pertinente, en función de los intereses y circunstancias personales de cada usuario (personalización máxima).



Esta situación imaginaria tiene ya su base real, materializada en los proyectos piloto realizados y en los grandes avances logrados para su creación en cuanto a estándares e infraestructura. Las principales empresas, como IBM, Microsoft, etc. participan activamente en su desarrollo, así como la comunidad investigadora, especialmente la universitaria. Por supuesto, el proyecto no hubiera sido posible sin el apoyo e impulso de la W3C, que junto con el sitio oficial www.semanticweb.org, se encarga de ofrecer toda la información disponible sobre los progresos en este ámbito. El interés por la WS se refleja en la celebración anual del Congreso internacional de la WWW, que en 2009 ha tenido lugar en la Universidad Politécnica de Madrid. También queda patente con la publicación de la revista Journal of Web Semantics.

En el terreno de las bibliotecas, la WS podrá ser decisiva de cara a la construcción de una Biblioteca Digital Universal, donde todo sea accesible de forma rápida y precisa, se encuentre donde se encuentre. Por supuesto, aún queda mucho camino por recorrer y la transición de la Web actual a la WS puede implicar un coste altísimo (en tiempo, dinero y esfuerzo), ya que no sólo se trata de estructurar la información web venidera, sino también la ya existente, labor que se prevé irrealizable.

\section{Trabajo futuro}

Durante la realización de este TFM se han encontrado temas, ejemplos o aplicaciones bastantes relacionados pero que no se han estudiado en el mismo, por este motivo, comentaremos algunos aspectos a profundizar en el futuro:

\begin{enumerate}
	
	\item 
	
	%\item Generar modelos para obtener la temperatura de una superficie a partir de la altitud y la localización del territorio.
	%\item Realizar un estudio con la función $terrain$ de R para la rugosidad del terreno, ya que dicha función no sólo permite obtener la pendiente y la orientación, sino otras variables cartográficas.
	%\item Ampliar el paquete desarrollado para facilitar este tipo de análisis, así como, aumentar el volumen de datos cargados en la misma, centrándonos en los MDT de España.
\end{enumerate}

%Adicionalmente, considero que este tema y el enfoque que le he dado puede tener interés para el público en general por lo que valoro la posibilidad de darle más difusión al trabajo.



