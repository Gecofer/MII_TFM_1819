% Conceptos Generales y Revisión de la Literatura

% En este capítulo debe ser proporcionado el estado del arte o referencial teórico sobre el tema al que se refiere el estudio. Un buen investigador no debe repetir trabajos ya concluidos o que ya están en marcha. Por eso esta sesión es donde el autor demuestra hasta dónde va la investigación actual en el campo de estudios en cuestión y establece las bases sobre las cuales desarrollará el estudio propuesto. También este punto se podría denominar "Situación actual de la tecnología". No se habla del estado actual de la tecnología con la que se desarrollará el trabajo, sino de qué otras aplicaciones que existen actualmente o han existido en el mercado (historia de la evolución tecnológica) y que realicen funcionalidades iguales o parecidas a las que se propone desarrollar en el TFG. En este punto hay que presentar lo que hay, simplemente describiendo. Se puede hacer alguna pequeña valoración que muestre los puntos fuertes de la tecnología o su utilidad, en que ámbitos se puede aplicar, para posteriormente, en el apartado de "Crítica al estado del arte" enjuiciar constructivamente cada aportación y de ahí extraer consecuencias para el TFG.

\chapter{Conceptos Geográficos}
\label{ch:sig}

\begin{quote}
  {\bf\textsc{Resumen:}} Este capítulo ...
\end{quote}


\section{Sistemas de Información Geográfica}


% Definición de los SIG


\subsection{Elementos de un SIG}


\subsection{Ejemplo de aplicación real de los SIG}



\section{?`Qué no es un SIG?}




\section{Tipos de SIG}


\subsection{Representación de datos geográficos}

- Modelo de datos Ráster
- Modelo de datos Vectorial

\subsection{Tipos de software SIG}


Hay varios SIG comerciales de escritorio: ESRI ArcGIS, Intergraph GeoMedia, Autodesk Au-toCAD, MapInfo Professional, Smallworld GIS y SuperMap.\\

Además, numerosos sistemas de software GIS de escritorio de fuente abierta también están disponibles para el manejo de datos geoespaciales: GRASS GIS, gvGIS, JUMP GIS, uDig, SAGA GIS, ILWIS, MapWindow GIS y QGIS.\\

Es improbable que todas las aplicaciones GIS utilicen el mismo software. Los diferentes proveedores tienen sus propios diseños de software propietario, modelos de datos y estructuras de almacenamiento de bases de datos. Por lo tanto, las bases de datos geográficas basadas en estos diseños no pueden comunicarse sin la conversión de datos. Para intercambiar información y compartir recursos de geo-bases de datos computacionales entre sistemas heterogéneos, se deben desarrollar herramientas de conversión para transferir datos de un formato a otro. Además, estas diversas estructuras de bases de datos SIG de escritorio hacen que el intercambio de datos remoto y el intercambio sean más difíciles debido a la accesibilidad limitada y la conversión de datos requerida.\\

Internet GIS o Web GIS crea un entorno único para compartir datos geoespaciales. Hay muchos programas de Internet GIS o Web GIS disponibles para compartir datos a través de la Web, como Esri’s ArcGIS Server, Intergraph’s GeoMedia WebMap, MapInfo’s MapExtreme, AutoDesk’s MapGuide, GE SmallWorld’s Internet Application Server y ER Mapper’s Image Web Server.\\

Aunque estos programas de Internet GIS ofrecen mejores herramientas para compartir datos en la Web, también tienen los problemas de los diseños de software propietario, los modelos de datos y las estructuras de almacenamiento de bases de datos. El intercambio de datos, facilitado por los avances en las tecnologías de red, se ve obstaculizado por la incompatibilidad de la variedad de modelos y formatos de datos utilizados en diferentes sitios.


\section{Resumen del capítulo}

