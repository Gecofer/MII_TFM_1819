%%%%%%%%%%%%%%%%%%%%%%%%%%%%% ANEXO %%%%%%%%%%%%%%%%%%%%%%%%%%%%%

%\section*{\centering{A – ANEXOS Y APÉNDICES }} % Añadir código
%\addcontentsline{toc}{section}{A - ANEXOS Y APÉNDICES}

% Los anexos y apéndices son materiales adicionales, utilizados para complementar el texto, añadidos al final del trabajo, con la finalidad de aclaración o de comprobación. Son elaborados por el autor y pretenden complementar una argumentación y sirven de fundamentación teórica, comprobación e ilustración (por ejemplo, mapas, leyes, códigos)

%\chapter*{Apendice A - }
%\label{ch:Apendice}

%\section*{\huge Apéndice A} 
%\section*{Fuentes de información para la descarga de MDT}
%\addcontentsline{toc}{chapter}{Apéndice A: Fuentes de información para descarga de MDT}

%\begin{itemize}
%	\item \item \url{https://www.cursosteledeteccion.com/fuentes-gratuitas-para-descargar-dem-modelo-de-elevacion-digital/}
%	\item \url{http://www.gisandbeers.com/descarga-de-dem-mundiales-mde/}
%	\item \url{https://gisgeography.com/free-global-dem-data-sources/}
%	\item \url{http://www.gpsvisualizer.com/elevation}
%	\item \url{https://mappinggis.com/2017/12/programas-gratuitos-para-trabajar-con-imagenes-de-satelite/}
%\end{itemize}

\newpage

\section*{\huge Apéndice A} 
\section*{Evolución de la Web}
\label{ch:ApendiceA}
\addcontentsline{toc}{chapter}{Apéndice A: Evolución de la Web}

\subsection{Tipos de Web} % Evolución de la Web

\section{Linked Data}






